\documentclass[11pt,a4paper]{article}
\usepackage[latin1]{inputenc}
\usepackage{amsmath}
\usepackage{amsfonts}
\usepackage{amssymb}
\usepackage{graphicx}
\usepackage{bussproofs}
\author{Alessandro Vasquez}

% Environments
\newtheorem{theorem}{Theorem}[section]
\newtheorem{lemma}[theorem]{Lemma}
\newtheorem{proposition}[theorem]{Proposition}
\newtheorem{corollary}[theorem]{Corollary}

\newenvironment{proof}[1][Proof]{\begin{trivlist}
		\item[\hskip \labelsep {\bfseries #1}]}{\end{trivlist}}
\newenvironment{definition}[1][Definition]{\begin{trivlist}
		\item[\hskip \labelsep {\bfseries #1}]}{\end{trivlist}}
\newenvironment{example}[1][Example]{\begin{trivlist}
		\item[\hskip \labelsep {\bfseries #1}]}{\end{trivlist}}
\newenvironment{remark}[1][Remark]{\begin{trivlist}
		\item[\hskip \labelsep {\bfseries #1}]}{\end{trivlist}}

\newcommand{\qed}{\nobreak \ifvmode \relax \else
	\ifdim\lastskip<1.5em \hskip-\lastskip
	\hskip1.5em plus0em minus0.5em \fi \nobreak
	\vrule height0.75em width0.5em depth0.25em\fi}

\newenvironment{bprooftree}
{\leavevmode\hbox\bgroup}
{\DisplayProof\egroup}
% End of environments

\title{Modelli della concorrenza - Formulario}
\begin{document}
	\maketitle
	\newpage
\section{Logica di Hoare}
\subsection{Regole di derivazione primitive}

\subsubsection{Istruzione vuota}
Sia data una formula proposizionale $\alpha$. Allora:
\begin{prooftree}
	\AxiomC{}
	\UnaryInfC{$\{\alpha\}\: skip \:\{\alpha\}$}
\end{prooftree}

\subsubsection{Assegnamento}
Sia data una formula proposizionale $\alpha$, una variabile $x$ e un'espressione $E$. Allora:
\begin{prooftree}
	\AxiomC{}
	\UnaryInfC{$\{\alpha \left[\frac{E}{x} \right]\}\ x := E \ \{\alpha\} $}
\end{prooftree}

\subsubsection{Conseguenza}
Siano date le formule proposizionali $p,\ q,\ p_1,\ q_1$ e il programma $P$. Allora:	
\[
\begin{bprooftree}
	\AxiomC{$ p_1 \rightarrow p $}
	\AxiomC{$ \{ p \}\ P\ \{ q \}$}
	\BinaryInfC{$ \{ p_1 \}\ P\ \{ q \}$}
\end{bprooftree}\qquad
\begin{bprooftree}
	\AxiomC{$ \{ p \}\ P\ \{ q \}$}
	\AxiomC{$ q \rightarrow q_1 $}
	\BinaryInfC{$ \{ p \}\ P\ \{ q_1 \}$}
\end{bprooftree}
\]
Dalle ultime due deduzioni si ottiene induttivamente la seguente:
\begin{prooftree}
	\AxiomC{$ p_1 \rightarrow p $}
	\AxiomC{$ \{ p \}\ P\ \{ q \}$}
	\AxiomC{$ q \rightarrow q_1 $}
	\TrinaryInfC{$ \{ p_1 \}\ P\ \{ q_1 \}$}
\end{prooftree}

\subsubsection{Sequenza}
Siano date le formule proposizionali $p,\ q,\ r$ e le istruzioni $C_1, \ C_2$. Allora:	
\begin{prooftree}
	\
	\AxiomC{$ \{ p \}\ C_1\ \{ q \}$}
	\AxiomC{$ \{ q \}\ C_2\ \{ r \}$}
	\BinaryInfC{$ \{ p \}\ C_1\, ;\, C_2\ \{ r \}  $}
\end{prooftree}

\subsubsection{Iterazione}
Siano dati un ciclo iterativo, la relativa condizione $B$, una sua invariante $i$, e il corpo del ciclo $C$. Allora:

\begin{prooftree}
	
	\AxiomC{$ \{ i \land B \}\ C\ \{ i \}$}
	\UnaryInfC{$ \{ i \}\ \tt while\ B\ do\ C\ od\rm \ \{ i \land \neg B \}$}
\end{prooftree}

\subsection{Regole di derivazione ottenibili induttivamente}
Siano date le formule proposizionali $p,\ q,\ r$ e il programma $P$. Allora:	

\begin{prooftree}
	\AxiomC{$ \{ p \}\ P\ \{ q \}$}
	\AxiomC{$ \{ r \}\ P\ \{ q \}$}
	\BinaryInfC{$ \{ p \lor r \}\ P\ \{ q \}$}
\end{prooftree}

Nel caso in cui $r = \neg p$, la precondizione nella conclusione della deduzione diventa una tautologia:

\begin{prooftree}
	\AxiomC{$ \{ p \}\ P\ \{ q \}$}
	\AxiomC{$ \{ \neg p \}\ P\ \{ q \}$}
	\BinaryInfC{$ \{ p \lor \neg p \}\ P\ \{ q \}$}
\end{prooftree}

\end{document}