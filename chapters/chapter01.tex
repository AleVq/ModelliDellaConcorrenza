% !TEX root = ../ModelliDellaConcorrenza.tex
\chapter{Logica di Hoare}
\label{chap:logicaHoare}

\section{Correttezza parziale}
Riprendendo le strutture delle triple, si vuole ora costruire un sistema formale basato sulla deduzione naturale e sui seguenti assiomi.
\subsection{Assiomi e regole di inferenza}

\paragraph{Istruzione vuota}
Sia data una formula proposizionale $\alpha$. Allora:
\begin{prooftree}
	\AxiomC{}
	\UnaryInfC{$\{\alpha\}\: skip \:\{\alpha\}$}
\end{prooftree}

\paragraph{Assegnamento}
Sia data una formula proposizionale $\alpha$, una variabile $x$ e un'espressione $E$. Allora:
\begin{prooftree}
	\AxiomC{}
	\UnaryInfC{$\{\alpha \left[\frac{E}{x} \right]\}\ x := E \ \{\alpha\} $}
\end{prooftree}

\paragraph{Conseguenza}
Siano date le formule proposizionali $p,\ q,\ p_1,\ q_1$ e il programma $P$. Allora:	
\[
\begin{bprooftree}
\AxiomC{$ p_1 \rightarrow p $}
\AxiomC{$ \{ p \}\ P\ \{ q \}$}
\BinaryInfC{$ \{ p_1 \}\ P\ \{ q \}$}
\end{bprooftree}\qquad
\begin{bprooftree}
\AxiomC{$ \{ p \}\ P\ \{ q \}$}
\AxiomC{$ q \rightarrow q_1 $}
\BinaryInfC{$ \{ p \}\ P\ \{ q_1 \}$}
\end{bprooftree}
\]
Dalle ultime due deduzioni si ottiene induttivamente la seguente:
\begin{prooftree}
	\AxiomC{$ p_1 \rightarrow p $}
	\AxiomC{$ \{ p \}\ P\ \{ q \}$}
	\AxiomC{$ q \rightarrow q_1 $}
	\TrinaryInfC{$ \{ p_1 \}\ P\ \{ q_1 \}$}
\end{prooftree}

\paragraph{Sequenza}
Siano date le formule proposizionali $p,\ q,\ r$ e le istruzioni $C_1, \ C_2$. Allora:	
\begin{prooftree}
	\
	\AxiomC{$ \{ p \}\ C_1\ \{ q \}$}
	\AxiomC{$ \{ q \}\ C_2\ \{ r \}$}
	\BinaryInfC{$ \{ p \}\ C_1\, ;\, C_2\ \{ r \}  $}
\end{prooftree}
La regola di sequenza \`e utile nei cicli il cui corpo presenta istruzioni diverse che coinvolgono le stesse variabili.

\paragraph{Iterazione}
Siano dati un ciclo iterativo, la relativa condizione $B$, una sua invariante $i$, e il corpo del ciclo $C$. Allora:

\begin{prooftree}
	
	\AxiomC{$ \{ i \land B \}\ C\ \{ i \}$}
	\UnaryInfC{$ \{ i \}\ \tt while\ B\ do\ C\ od\rm \ \{ i \land \neg B \}$}
\end{prooftree}

\subsection{Regole di derivazione ottenibili induttivamente}
Siano date le formule proposizionali $p,\ q,\ r$ e il programma $P$. Allora:	

\begin{prooftree}
	\AxiomC{$ \{ p \}\ P\ \{ q \}$}
	\AxiomC{$ \{ r \}\ P\ \{ q \}$}
	\BinaryInfC{$ \{ p \lor r \}\ P\ \{ q \}$}
\end{prooftree}

Nel caso in cui $r = \neg p$, la precondizione nella conclusione della deduzione diventa una tautologia:

\begin{prooftree}
	\AxiomC{$ \{ p \}\ P\ \{ q \}$}
	\AxiomC{$ \{ \neg p \}\ P\ \{ q \}$}
	\BinaryInfC{$ \{ p \lor \neg p \}\ P\ \{ q \}$}
\end{prooftree}

\paragraph{Controllo di flusso}
Siano date le formule proposizionali $p,\ q$. Siano dati inoltre una condizione $B$, un blocco $P$ (eseguito solo se $B$ \`e verificata) e un blocco $Q$ (eseguito altrimenti). Sia infine $R$ il blocco che include l'istruzione di controllo su $B$ e i blocchi $ P $ e $Q$. Vale la seguente deduzione:

\begin{prooftree}
	\AxiomC{$ \{ p \land B \}\ P\ \{ q \}$}
	\AxiomC{$ \{ p \land \neg B \}\ Q\ \{ q \}$}
	\BinaryInfC{$ \{ p \}\ R\ \{ q \}$}
\end{prooftree}
\newpage
\section{Correttezza totale}
Si vuole verificare la terminazione di un programma dato. La non-terminazione pu\`o avvenire solo in presenza di cicli: ci si concentrer\`a sullo studio di programmi iterativi.
Si supponga di avere un programma $P$ che presenta un ciclo while $W$ il cui corpo \`e denotato con $C$. \\
Si supponga di aver determinato un invariante $i$ del ciclo. Dimostrare che il $P$ termina significa determinare una espressione $E$ e un invariante $p$, indipendente da $E$ e da $i$, tali che:
\begin{enumerate}
	\item $ p \rightarrow E \geq 0$
	\item $ \vdash_p \{ p \land B \land E=K  \}\ C\ \{E < K\}   $
\end{enumerate}

\subsection{Definizioni e notazioni}
\begin{deff}
	Sia $V$ l'insieme delle variabili di un programma $P$. Uno stato $\sigma$ di P \`e una funzione $$\sigma: V \rightarrow \mathbb{Z}.$$
\end{deff}

\begin{deff}
	L'insieme di tutti gli stati di un programma \`e l'insieme: $$\Sigma \{\sigma:V\rightarrow \mathbb{Z}\}.$$
\end{deff}

\begin{deff}
	L'insieme di tutte le formule proposizionali costruite a partire da $V$ di chiama $\Pi$.
\end{deff}

Dati $\sigma \in \Sigma$ e $ p \in \Pi$, si dir\`a che la formula $p$ \`e valida nello stato $\sigma$ usando la notazione $$\sigma \models p.$$

\begin{deff}
	L'insieme di tutte le possibili asserzioni vere per lo stato $\sigma$ \`e l'insieme $$t(\sigma)=\{p \in \Pi : \sigma \models p\}.$$
\end{deff}

\begin{deff}
	L'insieme degli stati che rendono vera la formula $p$ \`e l'insieme $$m(p)=\{\sigma \in \Sigma : \sigma \models p \}.$$
	Tale insieme \`e definito come l'estensione della formula $p$.
\end{deff}

\begin{lemma}
	Siano $S \subseteq \Sigma $ e $F \subseteq \Pi.$ Allora 
	$$t(S)=\{p \in \Pi : \forall s \in S : \sigma \models p  \} = \bigcap_{\sigma \in S}t(\sigma),$$
	$$m(F)=\{\sigma \in \Sigma : \forall p \in F : \sigma \models p  \} = \bigcap_{p \in F}t(p).$$
\end{lemma}

\begin{lemma}
	Siano $A,B \in \Sigma$ tali che $A \subseteq B$. Allora $$ t(A) \supseteq t(B).$$
\end{lemma}

\begin{lemma}
	Siano $A,B \in \Pi$ tali che $A \subseteq B$. Allora $$ m(A) \supseteq m(B).$$
\end{lemma}

\begin{lemma}
	Siano $p,q$ formule proposizionali. Grazie alla definizione ricorsiva di formula proposizionale, valgono le seguenti propriet\`a:
	\begin{itemize}
		\item $m(\neg p) = \Sigma \setminus m(p)$
		\item $m(p \lor q) = m(p) \cup m(q)$
		\item $m(p \land q) = m(p) \cap m(q)$
		\item $(m \rightarrow q) = (\Sigma \setminus m(p) )\cup m(q)$
	\end{itemize}
	in cui l'ultima uguaglianza \`e data dal fatto che $p \rightarrow q \equiv \neg p \lor q$\footnote{Si veda la definizione di $\beta$-formula.}.
\end{lemma}

\subsection{Determinare la precondizione}
Si supponga di avere una tripla di Hoare priva della precondizione: $P\{q\}$. Si rende necessario trovare un metodo per determinarne la precondizione.\\
Tramite l'assioma dell'assegnamento \`e possibile ottenere la precondizione pi\`u debole possibile\footnote{La precondizione che pone meno vincoli allo stato iniziale del programma}. \\
\begin{es}
	Si supponga di avere la seguente tupla: $$C_1;C_2\{q\}.$$ 
	Tramite l'assioma dell'assegnamento applicato a $C_2\{q\}$ posso ottenere la tripla $$\vdash\{wp(C_2,q)\}\ C_2\ \{q\}\footnote{wp = weakest precondition}.$$
	Applicando poi l'assioma di assegnamento e l'assioma di sequenza a $C_1$, ottengo la tripla $$\{wp(C_1, wp(C_2,q))\}\ C_1;C_2\ \{q\},$$ nella quale compare la pi\`u debole precondizione della tupla iniziale.
\end{es}

\begin{es}
	\label{es:wpIter}
	Sia dato un generico programma iterativo $P\{q\}$:
	\begin{lstlisting}
	while B do
	C
	od
	\end{lstlisting}
	Si vuole determinare la precondizione pi\`u debole.
	Vi possono essere due possibili condizioni: $B$ \`e vera o no. 
	\begin{enumerate}
		\item Se $B$ \`e falsa prima del ciclo allora il while viene saltato e $P$ equivale a un'istruzione vuota. La precondizione in questo caso deve coincidere con la postcondizione: $$\{\neg B \land q \}.$$
		\item Se $B$ \`e vera prima del ciclo allora posso pensare di provare ad eseguire una volta il corpo del ciclo subito prima del ciclo stesso:
		\begin{lstlisting}
		C
		while B do
		C
		od
		\end{lstlisting}
		Posso quindi estrarre la precondizione pi\`u debole di $C$: $$wp(C, q) = (B \land wp(C;P,q)).$$
		Mettendo insieme le due precondizioni ottengo: $$wp(P, q)=\{(\neg B \land q )  \lor (B \land wp(C;P,q))\}.$$
	\end{enumerate}
\end{es}
Il risultato dell'esempio \ref{es:wpIter} potrebbe sembrare insoddisfacente: si \`e giunti a una formula pi\`u complessa di quella iniziale. \\
In effetti il calcolo di $wp$ per un programma iterativo non porta meccanicamente a un risultato: tale calcolo non definisce un algoritmo, la soluzione pu\`o essere determinata grazie a metodologie euristiche informalmente giustificate. Tuttavia, nei programmi non iterativi il calcolo di $wp$ pu\`o essere usato meccanicamente per determinare una soluzione.