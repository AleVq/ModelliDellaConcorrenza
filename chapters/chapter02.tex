% !TEX root = ../ModelliDellaConcorrenza.tex


\chapter{Introduzione}

Si vuole definire un formalismo per l'analisi di correttezza di programmi concorrenti.

\section{Sintassi}
Saranno ora definiti gli elementi di base del formalismo che si vuole definire.
	Siano $\mathcal{K}$\footnote{Talvota ci si riferir\`a a tale insieme come: $P_{CCS}.$} l'insieme di tutti i possibili processi e $Act$ l'insieme di tutte le possibili azioni: $$ Act=  \mathcal{A} \cup \overline{\mathcal{A}} \cup \{\tau\},$$ dove $\overline{\mathcal{A}} = \{\overline{a}\ |\ a \in \mathcal{A} \}$ \`e l'insieme delle coazioni di $a$.\\
\begin{deff}
	\label{def:ccsExpr}
	Si definisce espressione CCS una qualunque delle seguenti:
	\begin{itemize}
		\item \texttt{Nil} (il processo vuoto, denotato con $\emptyset$)
		\item $K$ (quale nome di un processo qualunque: $K \in \mathcal{K}$)
		\item $\alpha.P,$ dove $\alpha \in Act$
		\item $ \Sigma_{i \in I} P_i$
		\item $P_1\ |\ P_2$ (composizione parallela)
		\item $P \setminus L,$ ove $ L \subseteq A,$ ad esempio $P \setminus \{a, c\} $ significa che $P$ non pu\`o interagire con l'ambiente usando una qualunque tra le azioni $a, c, \overline{a}, \overline{c}$
		\item $P[f],$ dove $f$ \`e una funzione $f: Act \rightarrow Act$ tale che: $$f(\tau)=\tau,$$
		$$f(\overline{a})=\overline{f(a)} $$
	\end{itemize}
\end{deff}
La definizione \ref{def:ccsExpr} \`e ottenuta per induzione strutturale sui processi $P_i$.
\begin{lemma}
La coazione di una coazione $a$ \`e $a$ stessa: $\overline{(\overline{a})}=a.$
\end{lemma}

\begin{deff}
	Una specifica o programma CCS sar\`a un insieme di equazioni che specificano per ogni processo qual \`e l'espressione CCS che specifica il suo comportamento. L'insieme di equazioni sar\`a del tipo: $$K = P,\ K \in \mathcal{K}.$$
\end{deff}
Per ognuna di tali equazioni valgono le seguenti propriet\`a:
\begin{itemize}
	\item per ogni $K$ viene definita una sola equazione
	\item \`e ammessa la ricorsione, ad es. $K = a.\overline{b}.K$ che equivale a dire che il processo $K$ esegue $a$, poi $\overline{b}$ e poi si comporta come $K$ stesso
\end{itemize}

Si vuole definire ora un sistema di precedenza tra le operazioni. In questo corso vale il seguente sistema:
\begin{enumerate}
	\item $\setminus L;$
	\item $P[f];$
	\item $\alpha.P;$
	\item $|$;
	\item $+$.
\end{enumerate}
Le operazioni sono elencante per ordine decrescente di priorit\`a.
\begin{es}
	L'equazione $$ R + a.P\ |\ b.Q \setminus L$$ equivale all'equazione $$ R + (\ (a.P)\ |\ (b.(Q \setminus L))\ ).$$
\end{es}

\section{Semantica}
Si vuole passare dal sistema CCS precedentemente definito a un labelled transition systems (LTS). 
\subsection{Semantica operazionale strutturale}
Per introdurre un nuovo LTS, \`e necessario fissare un insieme di regole di inferenza, come \`e stato fatto nella logica di Hoare. Tali regole presentano espressioni CCS nelle loro proposizioni.
\subsubsection{Regole di base}

\begin{bprooftree}
	\AxiomC{}
	\UnaryInfC{$\alpha.P \xrightarrow{\alpha} P$}
\end{bprooftree}

\subsubsection{Regola della somma}
\begin{bprooftree}
	\AxiomC{$\alpha.P_i \xrightarrow{\alpha} P_j$}
	\RightLabel{$j \in I$}
	\UnaryInfC{$\sum_{i \in I}\ \alpha.P_i \xrightarrow{\alpha} P_j$}
\end{bprooftree}

\subsubsection{Regole della composizione parallela}
\[
\begin{bprooftree}
	\AxiomC{$P \xrightarrow{\alpha}P'$}
	\RightLabel{$\alpha \in Act$}
	\UnaryInfC{$P\ |\ Q \xrightarrow{\alpha} P'\ |\ Q$}
\end{bprooftree}\qquad
\begin{bprooftree}
	\AxiomC{$Q \xrightarrow{\alpha}Q'$}
	\RightLabel{$\alpha \in Act$}
	\UnaryInfC{$P\ |\ Q \xrightarrow{\alpha} P\ |\ Q'$}
\end{bprooftree}
\]
Mettendo insieme le ultime due, si ottiene la seguente:\\
\[
\begin{bprooftree}
		\AxiomC{$P \xrightarrow{\alpha}P'$}
		\AxiomC{$Q \xrightarrow{\alpha}Q'$}
		\RightLabel{$\alpha \in Act$}
		\BinaryInfC{$P\ |\ Q \xrightarrow{\alpha} P'\ |\ Q'$}
\end{bprooftree}
\]

\subsubsection{Regola della restrizione}
\begin{bprooftree}
	\AxiomC{$P \xrightarrow{\alpha} P'$}
	\RightLabel{$\alpha, \overline{\alpha} \notin L$}
	\UnaryInfC{$P \setminus L \xrightarrow{\alpha} P' \setminus L$}
\end{bprooftree}
\subsubsection{Regola delle etichettature}
\begin{bprooftree}
	\AxiomC{$P \xrightarrow{\alpha} P'$}
	\UnaryInfC{$P[f]  \xrightarrow{f(a)} P'[f] $}
\end{bprooftree}

\subsubsection{Regola della sostituzione}
\begin{bprooftree}
	\AxiomC{$P \xrightarrow{\alpha} P'$}
	\RightLabel{$P = K$}
	\UnaryInfC{$K \xrightarrow{\alpha} K' $}
\end{bprooftree}

\subsection{Equivalenza rispetto alle tracce}
\begin{deff}
	Dato un processo $P$, si definisce "tracce di P" l'insieme: $$Tracce(P)=\{w \in A_{CCS}^* : P \xrightarrow{w} P'  \}, $$ dove $ P \xrightarrow{w} P' $ equivale a dire che: $$ \exists\ P_1,..,P_n \in P_{CCS} : P \xrightarrow{x_1} P_1 \xrightarrow{x_2} P_2..\xrightarrow{x_n} P_n = P',$$
	con $w=x_1..x_n : x_i \in A_{CCS}.$\\
	Nel caso in cui $n=0$ avr\`o che $P=P'$.
\end{deff}
Due processi $P, Q$ si diranno \emph{equivalenti rispetto alle tracce} sse $Tracce(P)=Tracce(Q)$. Tale fatto \`e denotato come: $$P \thicksim^{T} Q.$$
Tale relazione di equivalenza non \`e sufficiente ai nostri scopi: non distinguendo le $tau$-transizioni tale relazione \`e troppo debole per i nostri scopi. Sar\`a necessario introdurre un'altra relazione.

\subsection{Bisimulazione}
\subsubsection{Bisimulazione forte}
\begin{deff}
	Sia $P \in P_{CCS}.$ Sia inoltre $\mathcal{R} \subseteq P_{CCS} \times P_{CCS} $ una relazione binaria. $\mathcal{R}$ \`e detta di bisimulazione forte sse $ \forall P, Q \in P_{CCS}, P\mathcal{R}Q$ sse:
	\begin{itemize}
		\item $ \forall \alpha \in A_{CCS}, P \xrightarrow{\alpha} P'$ allora $\exists Q' : Q \xrightarrow{\alpha} Q' \land P'\mathcal{R}Q',$
		\item $ \forall \alpha \in A_{CCS}, Q \xrightarrow{\alpha} Q'$ allora $\exists P' : P \xrightarrow{\alpha} P' \land P'\mathcal{R}Q'.$
	\end{itemize}
$P$ e $Q$ si diranno allora bisimili e si denoteranno nel seguente modo: $$P \thicksim^{Bis} Q.$$
\end{deff}
\begin{deff}
	La relazione $\thicksim^{Bis}$ \`e definita nel seguente modo: $$ \thicksim^{Bis} = \bigcup \{\mathcal{R}: \mathcal{R} \text{ \`e relazione di bisimulazione} \}.$$ $\thicksim^{Bis} $ \`e una relazione di equivalenza.
\end{deff}
Si pu\`o dire che due processi $P, Q \in P_{CCS} $ sono bisimili sse $\forall \alpha \in A_{CCS}:$
\begin{itemize}
	\item $\forall P': P \xrightarrow{\alpha}P', \exists Q': Q \xrightarrow{\alpha} Q' \land P' \thicksim^{Bis} Q',$
	\item $\forall Q': Q \xrightarrow{\alpha}Q', \exists P': P \xrightarrow{\alpha} P' \land P' \thicksim^{Bis} Q'.$
\end{itemize}
Si introducono ora alcune regole di equivalenza rispetto alla bisimulazione. \\
Siano $P, Q \in P_{CCS} \land P \thicksim^{Bis} Q,$ allora:
\begin{itemize}
	\item $\alpha.P \thicksim^{Bis} \alpha.Q,\ \forall \alpha \in A_{CCS}$
	\item $P + Z \thicksim^{Bis} Q + Z,\ \forall Z \in P_{CCS}$
	\item $P\ |\ Z \thicksim^{Bis} Q\ |\ Z,\ \forall Z \in P_{CCS}$
	\item $P \setminus L \thicksim^{Bis} Q \setminus L,\ \forall L \in A_{CCS}$
	\item $P[f] \thicksim^{Bis} Q[f],\ \forall f: \in A_{CCS} \rightarrow A_{CCS}$
\end{itemize}
Da tali regole \`e possibile concludere che $\thicksim^{Bis}$ \`e una relazione di congruenza.

\subsubsection{Bisimulazione debole}
Si supponga di voler astrarre la relazione di bisimulazione dalle $\tau$-transizioni. Ci\`o \`e necessario nell'ottica di astrarre l'osservazione delle transizioni da quelle non osservabili (appunto, le $\tau$-transizioni).\\
\`E necessario introdurre una regola che possa astrarre dalle $\tau$-transizioni. 
\begin{deff}
	La relazione $\Rightarrow$ \`e definita come segue: $$\forall \alpha \in A_{CCS},\ P \xRightarrow{\alpha} P' \text{ sse } P \xrightarrow{\tau^*} \xrightarrow{\alpha} \xrightarrow{\tau^*} P',\ \forall P, P' \in P_{CCS}.$$
\end{deff}
La relazione $'\Rightarrow'$ \`e dunque definita come segue: $$'\Rightarrow'\ \subseteq P_{CCS} \times A_{CCS} \times P_{CCS}.$$

\begin{deff}
	Le \emph{tracce osservabili} sono definite come l'insieme: $$L(P) = \{ w \in (A \cup \overline{A})^* : P \xRightarrow{w}\ \}.$$
\end{deff}

\begin{deff}
	Sia data una relazione $\mathcal{R} \subseteq P_{CCS} \times P_{CCS}. $ $\mathcal{R} $ \`e detta \emph{bisimulazione debole} sse:
	\begin{itemize}
		\item se $P \mathcal{R} Q, $ allora $\forall \alpha \in A_{CCS}$\footnote{Si ricorda che $\tau \in A_{CCS}.$} se $P \xrightarrow{\alpha} P', \text{ allora } \exists Q' : Q \xRightarrow{\alpha} Q' \land P' \mathcal{R} Q',$
		\item $ se Q \xrightarrow{\alpha} Q',$ allora $ P \xRightarrow{\alpha} \land P' \mathcal{R} Q'.$
	\end{itemize}
\end{deff}
\begin{lemma}
	Vale la seguente relazione: $$ P \xrightarrow{\tau} \land P \xRightarrow{\tau^*} P'.$$ \`E dunque possibile che $P'$ coincida con $P$.
\end{lemma}
Allora si dir\`a che $$P \approx^{Bis} Q \text{ sse } \exists \mathcal{R} \text{ di bisimulazione debole tale che } P \mathcal{R} Q.$$
\begin{deff}
	La relazione $ \approx^{Bis}$ pu\`o essere definita come segue: $$\approx^{Bis} = \bigcup \{\mathcal{R}\ |\ \mathcal{R} \text{ \`e relazione di bisimulazione debole}\}. $$
\end{deff}
Come per la bisimulazione forte, \`e possibile dire che due processi $P, Q \in P_{CCS}$ sono debolmente bisimili sse: 
\begin{itemize}
	\item $\forall \alpha \in A_{CCS},\ \forall P' : P \xrightarrow{\alpha}P',\ \exists Q': Q \xRightarrow{\alpha}Q' \land P' \approx^{Bis} Q',$
	\item $\forall \alpha \in A_{CCS},\ \forall Q' : Q \xrightarrow{\alpha}Q',\ \exists P': P \xRightarrow{\alpha}P' \land P' \approx^{Bis} Q'.$
\end{itemize}