% !TEX root = ../ModelliDellaConcorrenza.tex

\chapter*{Introduzione}
Sia dato un programma $P$. Ci si vuole convincere del suo buon funzionamento. Vi sono tre diverse strade che si posso intraprendere:
\begin{itemize}
	\item Sperimentazione
	\item Metodo assiomatico
	\item Model-checking
\end{itemize}
Questo corso tratter\`a le ultime due metodologie. Pi\`u precisamente, si divider\`a in tre parti:
\begin{enumerate}
	\item Correttezza dei programmi sequenziali
	\item Modelli della concorrenza
	\item Model-checking
\end{enumerate}
Si riporta ora un esempio che introduce all'applicazione del metodo assiomatico.
\begin{es}
	\label{es:001intro}
	Sia data la seguente funzione, ove $v$ \`e un vettore di interi e $n$ la sua dimensione:
	\begin{lstlisting}
		int f (int n, int v[]){
			int x = v[0];
			int h = 1;
			while (h < n){
				if (x < v[h]){
					x = v[h];
					h = h+1;
				}
			}
			return x;
		}
	\end{lstlisting}
A una prima analisi la funzione sembra ordinare il vettore in modo decrescente. Pi\`u formalmente, si pu\`o assumere che i requisiti della funzione siano i seguenti: 
	\begin{equation}
	\label{eq:requirements}
	\begin{array}{rcl}\forall i \in \{0,..,n-1\}\ :\ v[i] \leq x,\\
	\exists i \in \{0,..,n-1\}\ :\ x=v[i].\end{array}
	\end{equation}
	Si vuole dimostrare  tali requisiti per induzione sulla dimensione del vettore $v$. Sia data quindi la seguente ipotesi induttiva: 
	$$ Hp\ ind: \left\{ \begin{array}{rcl}
	 \forall i \in \{0,..,h-1\}\ :\ v[i] \leq x, \\
	\exists i \in \{0,..,h-1\}\ :\ x=v[i].
\end{array} \right. $$
	\`E necessario effettuare il passo induttivo e dimostrare tali propriet\`a per i casi base.
\end{es}

Formalizzando l'esempio \ref{es:001intro} si ottiene la tripla: $$\{n>0\}\ P\ \{\alpha\}.$$
Dove $n>1$ \`e chiamata precondizione, $P$ \`e il programma o la funzione e $\alpha$ \`e la postcondizione che consiste dei requisiti esplicitati in \eqref{eq:requirements}. \\
Tale formalismo sar\`a ripreso nel capitolo \ref{chap:logicaHoare}.